% Options for packages loaded elsewhere
\PassOptionsToPackage{unicode}{hyperref}
\PassOptionsToPackage{hyphens}{url}
\PassOptionsToPackage{dvipsnames,svgnames,x11names}{xcolor}
%
\documentclass[
  letterpaper,
  DIV=11,
  numbers=noendperiod]{scrreprt}

\usepackage{amsmath,amssymb}
\usepackage{lmodern}
\usepackage{iftex}
\ifPDFTeX
  \usepackage[T1]{fontenc}
  \usepackage[utf8]{inputenc}
  \usepackage{textcomp} % provide euro and other symbols
\else % if luatex or xetex
  \usepackage{unicode-math}
  \defaultfontfeatures{Scale=MatchLowercase}
  \defaultfontfeatures[\rmfamily]{Ligatures=TeX,Scale=1}
\fi
% Use upquote if available, for straight quotes in verbatim environments
\IfFileExists{upquote.sty}{\usepackage{upquote}}{}
\IfFileExists{microtype.sty}{% use microtype if available
  \usepackage[]{microtype}
  \UseMicrotypeSet[protrusion]{basicmath} % disable protrusion for tt fonts
}{}
\makeatletter
\@ifundefined{KOMAClassName}{% if non-KOMA class
  \IfFileExists{parskip.sty}{%
    \usepackage{parskip}
  }{% else
    \setlength{\parindent}{0pt}
    \setlength{\parskip}{6pt plus 2pt minus 1pt}}
}{% if KOMA class
  \KOMAoptions{parskip=half}}
\makeatother
\usepackage{xcolor}
\setlength{\emergencystretch}{3em} % prevent overfull lines
\setcounter{secnumdepth}{5}
% Make \paragraph and \subparagraph free-standing
\ifx\paragraph\undefined\else
  \let\oldparagraph\paragraph
  \renewcommand{\paragraph}[1]{\oldparagraph{#1}\mbox{}}
\fi
\ifx\subparagraph\undefined\else
  \let\oldsubparagraph\subparagraph
  \renewcommand{\subparagraph}[1]{\oldsubparagraph{#1}\mbox{}}
\fi

\usepackage{color}
\usepackage{fancyvrb}
\newcommand{\VerbBar}{|}
\newcommand{\VERB}{\Verb[commandchars=\\\{\}]}
\DefineVerbatimEnvironment{Highlighting}{Verbatim}{commandchars=\\\{\}}
% Add ',fontsize=\small' for more characters per line
\usepackage{framed}
\definecolor{shadecolor}{RGB}{241,243,245}
\newenvironment{Shaded}{\begin{snugshade}}{\end{snugshade}}
\newcommand{\AlertTok}[1]{\textcolor[rgb]{0.68,0.00,0.00}{#1}}
\newcommand{\AnnotationTok}[1]{\textcolor[rgb]{0.37,0.37,0.37}{#1}}
\newcommand{\AttributeTok}[1]{\textcolor[rgb]{0.40,0.45,0.13}{#1}}
\newcommand{\BaseNTok}[1]{\textcolor[rgb]{0.68,0.00,0.00}{#1}}
\newcommand{\BuiltInTok}[1]{\textcolor[rgb]{0.00,0.23,0.31}{#1}}
\newcommand{\CharTok}[1]{\textcolor[rgb]{0.13,0.47,0.30}{#1}}
\newcommand{\CommentTok}[1]{\textcolor[rgb]{0.37,0.37,0.37}{#1}}
\newcommand{\CommentVarTok}[1]{\textcolor[rgb]{0.37,0.37,0.37}{\textit{#1}}}
\newcommand{\ConstantTok}[1]{\textcolor[rgb]{0.56,0.35,0.01}{#1}}
\newcommand{\ControlFlowTok}[1]{\textcolor[rgb]{0.00,0.23,0.31}{#1}}
\newcommand{\DataTypeTok}[1]{\textcolor[rgb]{0.68,0.00,0.00}{#1}}
\newcommand{\DecValTok}[1]{\textcolor[rgb]{0.68,0.00,0.00}{#1}}
\newcommand{\DocumentationTok}[1]{\textcolor[rgb]{0.37,0.37,0.37}{\textit{#1}}}
\newcommand{\ErrorTok}[1]{\textcolor[rgb]{0.68,0.00,0.00}{#1}}
\newcommand{\ExtensionTok}[1]{\textcolor[rgb]{0.00,0.23,0.31}{#1}}
\newcommand{\FloatTok}[1]{\textcolor[rgb]{0.68,0.00,0.00}{#1}}
\newcommand{\FunctionTok}[1]{\textcolor[rgb]{0.28,0.35,0.67}{#1}}
\newcommand{\ImportTok}[1]{\textcolor[rgb]{0.00,0.46,0.62}{#1}}
\newcommand{\InformationTok}[1]{\textcolor[rgb]{0.37,0.37,0.37}{#1}}
\newcommand{\KeywordTok}[1]{\textcolor[rgb]{0.00,0.23,0.31}{#1}}
\newcommand{\NormalTok}[1]{\textcolor[rgb]{0.00,0.23,0.31}{#1}}
\newcommand{\OperatorTok}[1]{\textcolor[rgb]{0.37,0.37,0.37}{#1}}
\newcommand{\OtherTok}[1]{\textcolor[rgb]{0.00,0.23,0.31}{#1}}
\newcommand{\PreprocessorTok}[1]{\textcolor[rgb]{0.68,0.00,0.00}{#1}}
\newcommand{\RegionMarkerTok}[1]{\textcolor[rgb]{0.00,0.23,0.31}{#1}}
\newcommand{\SpecialCharTok}[1]{\textcolor[rgb]{0.37,0.37,0.37}{#1}}
\newcommand{\SpecialStringTok}[1]{\textcolor[rgb]{0.13,0.47,0.30}{#1}}
\newcommand{\StringTok}[1]{\textcolor[rgb]{0.13,0.47,0.30}{#1}}
\newcommand{\VariableTok}[1]{\textcolor[rgb]{0.07,0.07,0.07}{#1}}
\newcommand{\VerbatimStringTok}[1]{\textcolor[rgb]{0.13,0.47,0.30}{#1}}
\newcommand{\WarningTok}[1]{\textcolor[rgb]{0.37,0.37,0.37}{\textit{#1}}}

\providecommand{\tightlist}{%
  \setlength{\itemsep}{0pt}\setlength{\parskip}{0pt}}\usepackage{longtable,booktabs,array}
\usepackage{calc} % for calculating minipage widths
% Correct order of tables after \paragraph or \subparagraph
\usepackage{etoolbox}
\makeatletter
\patchcmd\longtable{\par}{\if@noskipsec\mbox{}\fi\par}{}{}
\makeatother
% Allow footnotes in longtable head/foot
\IfFileExists{footnotehyper.sty}{\usepackage{footnotehyper}}{\usepackage{footnote}}
\makesavenoteenv{longtable}
\usepackage{graphicx}
\makeatletter
\def\maxwidth{\ifdim\Gin@nat@width>\linewidth\linewidth\else\Gin@nat@width\fi}
\def\maxheight{\ifdim\Gin@nat@height>\textheight\textheight\else\Gin@nat@height\fi}
\makeatother
% Scale images if necessary, so that they will not overflow the page
% margins by default, and it is still possible to overwrite the defaults
% using explicit options in \includegraphics[width, height, ...]{}
\setkeys{Gin}{width=\maxwidth,height=\maxheight,keepaspectratio}
% Set default figure placement to htbp
\makeatletter
\def\fps@figure{htbp}
\makeatother
\newlength{\cslhangindent}
\setlength{\cslhangindent}{1.5em}
\newlength{\csllabelwidth}
\setlength{\csllabelwidth}{3em}
\newlength{\cslentryspacingunit} % times entry-spacing
\setlength{\cslentryspacingunit}{\parskip}
\newenvironment{CSLReferences}[2] % #1 hanging-ident, #2 entry spacing
 {% don't indent paragraphs
  \setlength{\parindent}{0pt}
  % turn on hanging indent if param 1 is 1
  \ifodd #1
  \let\oldpar\par
  \def\par{\hangindent=\cslhangindent\oldpar}
  \fi
  % set entry spacing
  \setlength{\parskip}{#2\cslentryspacingunit}
 }%
 {}
\usepackage{calc}
\newcommand{\CSLBlock}[1]{#1\hfill\break}
\newcommand{\CSLLeftMargin}[1]{\parbox[t]{\csllabelwidth}{#1}}
\newcommand{\CSLRightInline}[1]{\parbox[t]{\linewidth - \csllabelwidth}{#1}\break}
\newcommand{\CSLIndent}[1]{\hspace{\cslhangindent}#1}

\usepackage{booktabs}
\usepackage{longtable}
\usepackage{array}
\usepackage{multirow}
\usepackage{wrapfig}
\usepackage{float}
\usepackage{colortbl}
\usepackage{pdflscape}
\usepackage{tabu}
\usepackage{threeparttable}
\usepackage{threeparttablex}
\usepackage[normalem]{ulem}
\usepackage{makecell}
\usepackage{xcolor}
\KOMAoption{captions}{tableheading}
\makeatletter
\makeatother
\makeatletter
\@ifpackageloaded{bookmark}{}{\usepackage{bookmark}}
\makeatother
\makeatletter
\@ifpackageloaded{caption}{}{\usepackage{caption}}
\AtBeginDocument{%
\ifdefined\contentsname
  \renewcommand*\contentsname{Table of contents}
\else
  \newcommand\contentsname{Table of contents}
\fi
\ifdefined\listfigurename
  \renewcommand*\listfigurename{List of Figures}
\else
  \newcommand\listfigurename{List of Figures}
\fi
\ifdefined\listtablename
  \renewcommand*\listtablename{List of Tables}
\else
  \newcommand\listtablename{List of Tables}
\fi
\ifdefined\figurename
  \renewcommand*\figurename{Figure}
\else
  \newcommand\figurename{Figure}
\fi
\ifdefined\tablename
  \renewcommand*\tablename{Table}
\else
  \newcommand\tablename{Table}
\fi
}
\@ifpackageloaded{float}{}{\usepackage{float}}
\floatstyle{ruled}
\@ifundefined{c@chapter}{\newfloat{codelisting}{h}{lop}}{\newfloat{codelisting}{h}{lop}[chapter]}
\floatname{codelisting}{Listing}
\newcommand*\listoflistings{\listof{codelisting}{List of Listings}}
\makeatother
\makeatletter
\@ifpackageloaded{caption}{}{\usepackage{caption}}
\@ifpackageloaded{subcaption}{}{\usepackage{subcaption}}
\makeatother
\makeatletter
\@ifpackageloaded{tcolorbox}{}{\usepackage[many]{tcolorbox}}
\makeatother
\makeatletter
\@ifundefined{shadecolor}{\definecolor{shadecolor}{rgb}{.97, .97, .97}}
\makeatother
\makeatletter
\makeatother
\ifLuaTeX
  \usepackage{selnolig}  % disable illegal ligatures
\fi
\IfFileExists{bookmark.sty}{\usepackage{bookmark}}{\usepackage{hyperref}}
\IfFileExists{xurl.sty}{\usepackage{xurl}}{} % add URL line breaks if available
\urlstyle{same} % disable monospaced font for URLs
\hypersetup{
  pdftitle={Statistical Reasoning},
  pdfauthor={Wouter de Nooy; Sharon Klinkenberg},
  colorlinks=true,
  linkcolor={blue},
  filecolor={Maroon},
  citecolor={Blue},
  urlcolor={Blue},
  pdfcreator={LaTeX via pandoc}}

\title{Statistical Reasoning}
\author{Wouter de Nooy \and Sharon Klinkenberg}
\date{7/14/23}

\begin{document}
\maketitle
\ifdefined\Shaded\renewenvironment{Shaded}{\begin{tcolorbox}[boxrule=0pt, borderline west={3pt}{0pt}{shadecolor}, breakable, enhanced, interior hidden, frame hidden, sharp corners]}{\end{tcolorbox}}\fi

\renewcommand*\contentsname{Table of contents}
{
\hypersetup{linkcolor=}
\setcounter{tocdepth}{2}
\tableofcontents
}
\bookmarksetup{startatroot}

\hypertarget{introduction-and-readers-guide}{%
\chapter*{Introduction and Reader's
Guide}\label{introduction-and-readers-guide}}
\addcontentsline{toc}{chapter}{Introduction and Reader's Guide}

\markboth{Introduction and Reader's Guide}{Introduction and Reader's
Guide}

In the years that I have been teaching inferential statistics to
bachelor students in Communication Science, I have learned two things.
First, it is paramount that students thoroughly understand the
principles of statistical inference before they can apply statistical
inference correctly themselves. Second, formal notation, manual
calculation, and estimation details distract rather than help students
understand what they are doing. This book offers a non-technical but
thorough introduction to statistical inference. It discusses a minimal
set of concepts needed to understand both the possibilities and pitfalls
of estimation, null hypothesis testing, moderation, and mediation
analysis. It uses a minimum of formal notation.

\hypertarget{intended-audience-and-setting}{%
\section*{Intended Audience and
Setting}\label{intended-audience-and-setting}}
\addcontentsline{toc}{section}{Intended Audience and Setting}

\markright{Intended Audience and Setting}

This book is written as reading material for a follow-up course in
statistics, in the bachelor of Communication Science at the University
of Amsterdam. Students enrolled in this course have passe an
introductory course in statistics that explained how to change research
questions into variables and associations between variables, how to
select and execute the correct analysis or test (in SPSS) to answer
their research question, and how to interpret the results in a language
that is both comprehensible for the average reader and complying with
professional standards (APA standard for reporting test results). In
addition, they have learned the very basics of inferential statistics:
How to decide which null hypothesis to reject based on reported \emph{p}
values, and how to interpret confidence intervals.

This book is meant for use in a flipped-classroom setting. Students
should read the text, watch embedded videos, and play with the
interactive content before they meet in class. Class meetings are used
to answer questions raised by the students, do group work to exercise
with the concepts and techniques presented in the text, and do short
tests to check understanding.

\hypertarget{interactive-content}{%
\section*{Interactive Content}\label{interactive-content}}
\addcontentsline{toc}{section}{Interactive Content}

\markright{Interactive Content}

The interactive content in this book replaces simulations that used to
be demonstrated during lectures. I expect that doing simulations
yourself rather than watching them being done by someone else enhances
understanding. I have tried to break down the simulations into smaller
steps, confronting the student several times with essentially the same
simulation, but with added complexity. I hope that this approach
enhances understanding and remembrance and, at the same time, avoids
frustration caused by complex dashboards offering all options at once.

\hypertarget{disclaimer}{%
\section*{Disclaimer}\label{disclaimer}}
\addcontentsline{toc}{section}{Disclaimer}

\markright{Disclaimer}

The example data sets have been generated for the purpose of
demonstrating statistical techniques. These are not real data and no
conclusions should be drawn from the results obtained from the data.

\hypertarget{acknowledgements}{%
\section*{Acknowledgements}\label{acknowledgements}}
\addcontentsline{toc}{section}{Acknowledgements}

\markright{Acknowledgements}

Adam Sasiadek developed the more complicated Shiny apps in this book.
The College of Communication at the University of Amsterdam generously
supported the creation of the apps whereas this university's Grassroots
Project for ICT in education refused to support it. Renske van Bronswijk
corrected my English. Any remaining errors result from changes and
additions that I applied afterwards. My colleague Peter Neijens
commented on a draft of this text. Among the first tutors using this
book, Chei Billedo, Marcel van Egmond, Matthijs Elenbaas, Andreas
Goldberg, Bregje van Groningen, Luzia Helfer, Rhianne Hoek, Laura
Jacobs, Jeroen Jonkman, Fam te Poel, Sanne Schinkel, Christin Scholz,
Ragnheiður Torfadóttir, Philipp Mendoza, and she whose name I am not
allowed to mention offered many suggestions that improved the text
substantially. Among the students who helped to improve the book, Alissa
Hilbertz stands out for her careful language corrections and
suggestions.

This work is licensed under a Creative Commons Attribution-ShareAlike
4.0 International License.

\part{Sampling Distribution: How Different Could My Sample Have Been?}

\begin{quote}
Key concepts: inferential statistics, generalization, population, random
sample, sample statistic, sampling space, random variable, sampling
distribution, probability, probability distribution, discrete
probability distribution, expected value/expectation, unbiased
estimator, parameter, (downward) biased, representative sample,
continuous variable, continuous probability distribution, probability
density, (left-hand/right-hand) probability.
\end{quote}

Watch this micro lecture on sampling distributions for an overview of
the chapter.

\begin{verbatim}
PhantomJS not found. You can install it with webshot::install_phantomjs(). If it is installed, please make sure the phantomjs executable can be found via the PATH variable.
\end{verbatim}

\hypertarget{summary}{%
\section*{Summary}\label{summary}}
\addcontentsline{toc}{section}{Summary}

\markright{Summary}

\BeginKnitrBlock{rmdimportant}

What does our sample tell us about the population from which it was
drawn? \EndKnitrBlock{rmdimportant}

Statistical inference is about estimation and null hypothesis testing.
We have collected data on a random sample and we want to draw
conclusions (make inferences) about the population from which the sample
was drawn. From the proportion of yellow candies in our sample bag, for
instance, we want to estimate a plausible range of values for the
proportion of yellow candies in a factory's stock (confidence interval).
Alternatively, we may want to test the null hypothesis that one fifth of
the candies in a factory's stock is yellow.

The sample does not offer a perfect miniature image of the population.
If we would draw another random sample, it would have different
characteristics. For instance, it would contain more or fewer yellow
candies than the previous sample. To make an informed decision on the
confidence interval or null hypothesis, we must compare the
characteristic of the sample that we have drawn to the characteristics
of the samples that we could have drawn.

The characteristics of the samples that we could have drawn constitute a
sampling distribution. Sampling distributions are the central element in
estimation and null hypothesis testing. In this chapter, we simulate
sampling distributions to understand what they are. Here,
\emph{simulation} means that we let a computer draw many random samples
from a population.

In Communication Science, we usually work with samples of human beings,
for instance, users of social media, people looking for health
information or entertainment, citizens preparing to cast a political
vote, an organization's stakeholders, or samples of media content such
as tweets, tv advertisements, or newspaper articles. In the current and
two subsequent chapters, however, we avoid the complexities of these
samples.

We focus on a very tangible kind of sample, namely a bag of candies,
which helps us understand the basic concepts of statistical inference:
sampling distributions (the current chapter), probability distributions
(Chapter @ref(probmodels)), and estimation (Chapter @ref(param-estim)).
Once we thoroughly understand these concepts, we turn to Communication
Science examples.

\hypertarget{statistical-inference-making-the-most-of-your-data}{%
\chapter{Statistical Inference: Making the Most of Your
Data}\label{statistical-inference-making-the-most-of-your-data}}

Statistics is a tool for scientific research. It offers a range of
techniques to check whether statements about the observable world are
supported by data collected from that world. Scientific theories strive
for general statements, that is, statements that apply to many
situations. Checking these statements requires lots of data covering all
situations addressed by theory.

Collecting data, however, is expensive, so we would like to collect as
little data as possible and still be able to draw conclusions about a
much larger set. The cost and time involved in collecting large sets of
data are also relevant to applied research, such as market research. In
this context we also like to collect as little data as necessary.

\emph{Inferential statistics} offers techniques for making statements
about a larger set of observations from data collected for a smaller set
of observations. The large set of observations about which we want to
make a statement is called the \emph{population}. The smaller set is
called a \emph{sample}. We want to \emph{generalize} a statement about
the sample to a statement about the population from which the sample was
drawn.

Traditionally, statistical inference is generalization from the data
collected in a \emph{random sample} to the population from which the
sample was drawn. This approach is the focus of the present book because
it is currently the most widely used type of statistical inference in
the social sciences. We will, however, point out other approaches in
Chapter @ref(crit-discus).

Statistical inference is conceptually complicated and for that reason
quite often used incorrectly. We will therefore spend quite some time on
the principles of statistical inference. Good understanding of the
principles should help you to recognize and avoid incorrect use of
statistical inference. In addition, it should help you to understand the
controversies surrounding statistical inference and developments in the
practice of applying statistical inference that are taking place.
Investing time and energy in fully understanding the principles of
statistical inference really pays off later.

\bookmarksetup{startatroot}

\hypertarget{summary-1}{%
\chapter{Summary}\label{summary-1}}

In summary, this book has no content whatsoever.

\begin{Shaded}
\begin{Highlighting}[]
\DecValTok{1} \SpecialCharTok{+} \DecValTok{1}
\end{Highlighting}
\end{Shaded}

\begin{verbatim}
[1] 2
\end{verbatim}

\bookmarksetup{startatroot}

\hypertarget{references}{%
\chapter*{References}\label{references}}
\addcontentsline{toc}{chapter}{References}

\markboth{References}{References}

\hypertarget{refs}{}
\begin{CSLReferences}{0}{0}
\end{CSLReferences}



\end{document}
